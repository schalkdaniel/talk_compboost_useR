\documentclass[10pt]{beamer}\usepackage[]{graphicx}\usepackage[]{color}
%% maxwidth is the original width if it is less than linewidth
%% otherwise use linewidth (to make sure the graphics do not exceed the margin)
\makeatletter
\def\maxwidth{ %
  \ifdim\Gin@nat@width>\linewidth
    \linewidth
  \else
    \Gin@nat@width
  \fi
}
\makeatother

\definecolor{fgcolor}{rgb}{0.345, 0.345, 0.345}
\newcommand{\hlnum}[1]{\textcolor[rgb]{0.686,0.059,0.569}{#1}}%
\newcommand{\hlstr}[1]{\textcolor[rgb]{0.192,0.494,0.8}{#1}}%
\newcommand{\hlcom}[1]{\textcolor[rgb]{0.678,0.584,0.686}{\textit{#1}}}%
\newcommand{\hlopt}[1]{\textcolor[rgb]{0,0,0}{#1}}%
\newcommand{\hlstd}[1]{\textcolor[rgb]{0.345,0.345,0.345}{#1}}%
\newcommand{\hlkwa}[1]{\textcolor[rgb]{0.161,0.373,0.58}{\textbf{#1}}}%
\newcommand{\hlkwb}[1]{\textcolor[rgb]{0.69,0.353,0.396}{#1}}%
\newcommand{\hlkwc}[1]{\textcolor[rgb]{0.333,0.667,0.333}{#1}}%
\newcommand{\hlkwd}[1]{\textcolor[rgb]{0.737,0.353,0.396}{\textbf{#1}}}%
\let\hlipl\hlkwb

\usepackage{framed}
\makeatletter
\newenvironment{kframe}{%
 \def\at@end@of@kframe{}%
 \ifinner\ifhmode%
  \def\at@end@of@kframe{\end{minipage}}%
  \begin{minipage}{\columnwidth}%
 \fi\fi%
 \def\FrameCommand##1{\hskip\@totalleftmargin \hskip-\fboxsep
 \colorbox{shadecolor}{##1}\hskip-\fboxsep
     % There is no \\@totalrightmargin, so:
     \hskip-\linewidth \hskip-\@totalleftmargin \hskip\columnwidth}%
 \MakeFramed {\advance\hsize-\width
   \@totalleftmargin\z@ \linewidth\hsize
   \@setminipage}}%
 {\par\unskip\endMakeFramed%
 \at@end@of@kframe}
\makeatother

\definecolor{shadecolor}{rgb}{.97, .97, .97}
\definecolor{messagecolor}{rgb}{0, 0, 0}
\definecolor{warningcolor}{rgb}{1, 0, 1}
\definecolor{errorcolor}{rgb}{1, 0, 0}
\newenvironment{knitrout}{}{} % an empty environment to be redefined in TeX

\usepackage{alltt}

%% include header:
\input{./header}

%% include template:
\input{./templates/metropolis_cert}



%% kableExtra stuff:
%% ----------------------------------------

\usepackage{booktabs}
\usepackage{longtable}
\usepackage{array}
\usepackage{multirow}
\usepackage{wrapfig}
\usepackage{float}
\usepackage{colortbl}
\usepackage{pdflscape}
\usepackage{tabu}
\usepackage{threeparttable}
\usepackage{threeparttablex}
\usepackage[normalem]{ulem}
\usepackage{makecell}
\usepackage{xcolor}


%% Title:
%% ----------------------------------------

\title{compboost}
\subtitle{Fast and Flexible Way of bla}
\date{\today}
\author{Daniel Schalk}
\institute{LMU Munich\\Working Group Computational Statistics}

%% Wrap Shaded around Shunk to have a nices R output:
%% --------------------------------------------------

\let\Oldkframe\kframe
\let\endOldkframe\endkframe

\renewenvironment{kframe}
 {\scriptsize\definecolor{shadecolor}{RGB}{240,240,240}\begin{Shaded}\Oldkframe}
 {\endOldkframe\end{Shaded}\normalsize}

%% Prevent code from printing over margin:
%% --------------------------------------------------



%% Content:
%% ----------------------------------------
\IfFileExists{upquote.sty}{\usepackage{upquote}}{}
\begin{document}







%% SETUP Chunk:
%% ----------------------------------------



\maketitle



\section{Use-Case}

\begin{frame}{Use-Case}

\begin{itemize}
  \item
    We own a small booth at the city center of Toulouse that sells beer.

  \item
    As we are very interested in our customers' health, we only sell to customers who we expect to drink less than 110 liters per year.

  \item
    To estimate how much a customer drinks, we have collected data from 200 customers in recent years.

  \item
    These data include the beer consumption (in liter), age, sex, country of origin, weight, body size, and 200 characteristics gained from app usage (that have absolutely no influence).
\end{itemize}

\end{frame}

\begin{frame}[fragile]{Use-Case}

\begin{knitrout}
\definecolor{shadecolor}{rgb}{0.969, 0.969, 0.969}\color{fgcolor}\begin{table}[H]
\centering
\resizebox{\linewidth}{!}{
\begin{tabular}{rllrrrrr}
\toprule
\textbf{beer\_consumption} & \textbf{gender} & \textbf{country} & \textbf{age} & \textbf{weight} & \textbf{height} & \textbf{app\_usage1} & \textbf{app\_usage2}\\
\midrule
\rowcolor{gray!6}  106.5 & m & Seychelles & 33 & 87.17 & 172.9 & 0.1680 & 0.6063\\
85.5 & f & Seychelles & 52 & 89.38 & 200.4 & 0.8075 & 0.9376\\
\rowcolor{gray!6}  116.5 & f & Czechia & 54 & 92.03 & 178.7 & 0.3849 & 0.2644\\
67.0 & m & Australia & 32 & 63.53 & 186.3 & 0.3277 & 0.3801\\
\rowcolor{gray!6}  43.0 & f & Australia & 51 & 64.73 & 175.0 & 0.6021 & 0.8075\\
\addlinespace
85.0 & m & Austria & 43 & 95.74 & 173.2 & 0.6044 & 0.9781\\
\rowcolor{gray!6}  79.0 & f & Austria & 55 & 87.65 & 156.3 & 0.1246 & 0.9579\\
107.0 & f & Austria & 24 & 93.17 & 161.4 & 0.2946 & 0.7627\\
\rowcolor{gray!6}  57.0 & m & USA & 55 & 76.27 & 182.5 & 0.5776 & 0.5096\\
89.0 & m & USA & 16 & 72.21 & 203.3 & 0.6310 & 0.0645\\
\bottomrule
\end{tabular}}
\end{table}


\end{knitrout}

\end{frame}


\begin{frame}{Use-Case}

With these data we want to answer the following questions:

\begin{itemize}
  \item
    Which of the customers' characteristics are important to be able to determine the consumption?
  \item
    How does the effect of important features look like?
  \item
    How does the model behave on unseen data?
\end{itemize}

\end{frame}




\section{What is Component-Wise Boosting?}

\begin{frame}{The General Idea}

\begin{center}
\includegraphics[width=\textwidth]{./images/cboost_gif_norisk.png}
\end{center}

\end{frame}

\begin{frame}{}

  \begin{itemize}

    \item
      Inherent (unbiased) feature selection.

    \item
      Resulting model is sparse since important effects are selected first and therefore it is able to learn in high-dimensional feature spaces ($p \gg n$).

    \item
      Parameters are updated iteratively. Therefore, the whole trace of how the model evolves is available.

  \end{itemize}

\end{frame}




\section{The Idea Behind Compboost}

\begin{frame}{About Compboost}


The \texttt{compboost} package is a fast and flexible framework for model-based boosting:

\begin{itemize}

  \item
    With \texttt{mboost} as standard, we want to keep the modular principle of defining custom base-learner and losses.

  \item
    Completely written in \texttt{C++} and exposed by \texttt{Rcpp} to obtain high performance and full memory control.

  \item
    \texttt{R} API is written in \texttt{R6} to provide convenient wrapper.

  \item
    Major parts of the \texttt{compboost} functionality are unit tested against \texttt{mboost} to ensure correctness.

\end{itemize}

\end{frame}


\begin{frame}{Runtime and Memory Considerations}

\begin{itemize}

  \item
    Special data structures are used to store matrices as sparse matrix or to speed up the algorithm by reducing the number of repetitive calculations.

  \item
    Optimizer are parallelized via openmp (figure was made by training 5000 iteration with just spline base-learners):
    \begin{center}
    \includegraphics[width=0.7\textwidth]{./images/cboost_runtime.pdf}
    \end{center}



\end{itemize}

\end{frame}


\section{A Short Demonstration}

\begin{frame}[fragile]{Using Convenience Wrapper}

\begin{knitrout}
\definecolor{shadecolor}{rgb}{0.969, 0.969, 0.969}\color{fgcolor}\begin{kframe}
\begin{alltt}
\hlkwd{set.seed}\hlstd{(}\hlnum{618}\hlstd{)}
\hlstd{cboost} \hlkwb{=} \hlkwd{boostSplines}\hlstd{(}\hlkwc{data} \hlstd{= beer_data,} \hlkwc{target} \hlstd{=} \hlstr{"beer_consumption"}\hlstd{,}
  \hlkwc{loss} \hlstd{= LossAbsolute}\hlopt{$}\hlkwd{new}\hlstd{(),} \hlkwc{learning_rate} \hlstd{=} \hlnum{0.1}\hlstd{,} \hlkwc{iterations} \hlstd{=} \hlnum{5000L}\hlstd{,}
  \hlkwc{penalty} \hlstd{=} \hlnum{10}\hlstd{,} \hlkwc{oob_fraction} \hlstd{=} \hlnum{0.3}\hlstd{,} \hlkwc{trace} \hlstd{=} \hlnum{2500L}\hlstd{)}
\end{alltt}
\begin{verbatim}
##    1/5000   risk = 24  oob_risk = 24   
## 2500/5000   risk = 0.6  oob_risk = 8.3   
## 5000/5000   risk = 0.44  oob_risk = 8.3   
## 
## 
## Train 5000 iterations in 19 Seconds.
## Final risk based on the train set: 0.44
\end{verbatim}
\end{kframe}
\end{knitrout}

\end{frame}

\begin{frame}[fragile]{Visualizing Results}

\begin{knitrout}
\definecolor{shadecolor}{rgb}{0.969, 0.969, 0.969}\color{fgcolor}\begin{kframe}
\begin{alltt}
\hlstd{gg1} \hlkwb{=} \hlstd{cboost}\hlopt{$}\hlkwd{plotInbagVsOobRisk}\hlstd{()}
\hlstd{gg2} \hlkwb{=} \hlstd{cboost}\hlopt{$}\hlkwd{plotFeatureImportance}\hlstd{()}

\hlstd{gridExtra}\hlopt{::}\hlkwd{grid.arrange}\hlstd{(gg1, gg2,} \hlkwc{ncol} \hlstd{=} \hlnum{2L}\hlstd{)}
\end{alltt}
\end{kframe}
\includegraphics[width=\textwidth]{figure/unnamed-chunk-6-1} 

\end{knitrout}

\end{frame}

\begin{frame}[fragile]{Visualizing Results}

\begin{knitrout}
\definecolor{shadecolor}{rgb}{0.969, 0.969, 0.969}\color{fgcolor}\begin{kframe}
\begin{alltt}
\hlstd{cboost}\hlopt{$}\hlkwd{train}\hlstd{(}\hlnum{2000L}\hlstd{)}

\hlstd{gg1} \hlkwb{=} \hlstd{cboost}\hlopt{$}\hlkwd{plotFeatureImportance}\hlstd{()}
\hlstd{gg2} \hlkwb{=} \hlstd{cboost}\hlopt{$}\hlkwd{plot}\hlstd{(}\hlstr{"age_spline"}\hlstd{,} \hlkwc{iters} \hlstd{=} \hlkwd{c}\hlstd{(}\hlnum{50}\hlstd{,} \hlnum{100}\hlstd{,} \hlnum{500}\hlstd{,} \hlnum{1000}\hlstd{,} \hlnum{2000}\hlstd{,} \hlnum{4000}\hlstd{))}

\hlstd{gridExtra}\hlopt{::}\hlkwd{grid.arrange}\hlstd{(gg1, gg2,} \hlkwc{ncol} \hlstd{=} \hlnum{2L}\hlstd{)}
\end{alltt}
\end{kframe}
\includegraphics[width=\textwidth]{figure/unnamed-chunk-7-1} 

\end{knitrout}


\end{frame}

\begin{frame}{More Advanced Customizations}

\begin{itemize}

  \item
    Custom loss function and base-learner

  \item
    Advanced stopper for early stopping (e.g. time or performance based stopping)

  \item
    Parallelization via openmp is controlled by the optimizer, e.g. \texttt{OptimizerCoordinateDescent\$new(4L)}

\end{itemize}

\end{frame}


\section{What's Next?}

\begin{frame}{What's Next?}

\begin{itemize}
  \item
    Research on computational aspects of the algorithm:
    \begin{itemize}
      \item
        More stable base-learner selection process via resampling
      \item
        Base-learner selection for arbitrary performance measures
      \item
        Smarter and faster optimizer to select base-learner
    \end{itemize}

  \item
    Greater functionality:
    \begin{itemize}
      \item
        Functional data structures and loss functions
      \item
        Unbiased feature selection
      \item
        Effect decomposition into constant, linear, and non-linear
    \end{itemize}

    \item
      Reducing the memory load by applying binning to numerical features.

    \item
      Python API
\end{itemize}

\end{frame}



\begin{frame}[plain, standout]
  Questions?
\end{frame}


\end{document}
